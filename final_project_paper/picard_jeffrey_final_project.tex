\documentclass[12pt]{article}

\usepackage{epsfig}
\usepackage{rotating}
\usepackage{lscape}
\usepackage{amsfonts}
\usepackage{amssymb}
\usepackage{theorem}
\usepackage{amsmath}
\usepackage{graphicx}
\usepackage{citesort}
\usepackage{float}
\usepackage{color}
\usepackage{array}
\usepackage{color}
\usepackage{array}
\usepackage{float}
\usepackage{url}
\usepackage{setspace}
%\usepackage{amsmath,amsfonts,amssymb,cite,array,graphicx}

\textwidth=17.36cm
\textheight=23.90cm
\hoffset=-2.37cm
\voffset=-2.5cm
\parskip=1pt

\renewcommand{\rmdefault}{ptm}

\newtheorem{definition}{Definition}[section]
\newtheorem{theorem}{Theorem}[section]
\newtheorem{corollary}{Corollary}[section]
\newtheorem{lemma}{Lemma}[section]
\newtheorem{remark}{Remark}[section]


\date{\small\today} \title{CS730 Final Project} \author{Jeffrey Picard} 
\doublespacing

\begin{document}

\maketitle


\begin{abstract}
In this paper two different methods of speeding up A* search through symmetry reduction are explored. Some considerations for implementation are mentioned,
and it is shown that each method may be extended to slightly larger set of problems than previously considered.
\end{abstract}


% main text
\section{Introduction}
The two methods that were explored are Rectangular Symmetry Reduction (RSR) and Jump Point Symmetries (JPS). RSR identifies large empty spaces
in the search space and removes them leaving only the exterior nodes and inserting ``macro edges'' so an agent may jump across the space in order to preserve
optimality. JPS by contrast in an online method that is done by recursively pruning all neighbors of the current node that can be optimally reached by the 
node's parent without going through the current node. The recursion ends when a node is found that cannot be reached optimally by the current node's parent. 
Therefore the optimal path to it is through the current node. JPS as it was first introduced in\cite{Har2011} was only stipulated to work for a single goal.
In this paper it is shown that this was easily also be extended to multiple goals and optimality is maintained. Practical considerations for the implementation
of these algorithms are also discussed.
	
\section{Implementation}
This problem is interesting because both RSR and JPS have shown encouraging results in increasing the speed of A* and they preserve optimality. In addition to
this both of these methods are not inherently mutually exclusive to other methods of increasing the performance of A* such as ``swaps''  and the ``dead-end
heuristic''\cite{Har2010,Har2011}.  Therefore an even greater performance increase could be possible by combining these methods with existing ones. 
Even though JPS showed better overall performance than RSR, RSR is also interesting in that it raises the question of the possibility of other ways to do 
preprocessing on a map in order to decrease search time. 

\section{Results}
I will address this problem by implementing both RSR and JPS and comparing their performance with that of normal A*. The implementation of the two symmetry
breaking methods will be the main portion of the project and it will be tested on maps from movingai.com. An additional piece that would be interesting to 
look at is implementing 8-way movement and seeing if JPS fairs better or worse in that context (RSR, being done mostly offline shouldn't see much of a 
difference in whether or note 8-way movement is allowed). Also implementing other methods for speeding up A*, such as hierarchical methods or another method of
removing unnecessary paths like ``swaps'' or the ``dead-end heuristic'', and seeing how they do in conjunction with RSR and JPS could provide insight into 
the feasibility of actually combining them with other methods.

\section{Conslusion}
I will have plots showing the speed up factor vs path length for each of the different maps I test on. The graphs will show both time and node expansions.

\section{Further Work}
This will be a solo project, so I will be doing all of the work myself.

\bibliographystyle{elsarticle-num}
\bibliography{main_ai}


\end{document}
